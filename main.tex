%%%%%%%%%%%%%%%%%%%%%%%%%%%%%%%%%%%
%This is the LaTeX ARTICLE template for RSC journals
%Copyright The Royal Society of Chemistry 2016
%%%%%%%%%%%%%%%%%%%%%%%%%%%%%%%%%%%

\documentclass[twoside,twocolumn,9pt]{article}
\usepackage{extsizes}
\usepackage[super,sort&compress,comma]{natbib} 
\usepackage[version=3]{mhchem}
\usepackage[left=1.5cm, right=1.5cm, top=1.785cm, bottom=2.0cm]{geometry}
\usepackage{balance}
\usepackage{mathptmx}
\usepackage{sectsty}
\usepackage{graphicx} 
\usepackage{lastpage}
\usepackage[format=plain,justification=justified,singlelinecheck=false,font={stretch=1.125,small,sf},labelfont=bf,labelsep=space]{caption}
\usepackage{float}
\usepackage{fancyhdr}
\usepackage{fnpos}
\usepackage[english]{babel}
\addto{\captionsenglish}{%
  \renewcommand{\refname}{Notes and references}
}
\usepackage{array}
\usepackage{droidsans}
\usepackage{charter}
\usepackage[T1]{fontenc}
\usepackage[usenames,dvipsnames]{xcolor}
\usepackage{setspace}
\usepackage[compact]{titlesec}
\usepackage{hyperref}
%%%Please don't disable any packages in the preamble, as this may cause the template to display incorrectly.%%%


\usepackage{balance}
\usepackage{times,mathptmx}
\usepackage{sectsty}
\usepackage{graphicx} 
\usepackage{lastpage}
\usepackage[format=plain,justification=justified,singlelinecheck=false,font={stretch=1.125,small,sf},labelfont=bf,labelsep=space]{caption}
\usepackage{float}
\usepackage{fancyhdr}
\usepackage{fnpos}
\usepackage[english]{babel}
\addto{\captionsenglish}{%
  \renewcommand{\refname}{Notes and references}
}
\usepackage{array}
\usepackage{droidsans}
\usepackage{charter}
\usepackage[T1]{fontenc}
\usepackage[usenames,dvipsnames]{xcolor}
\usepackage{setspace}
\usepackage[compact]{titlesec}
\usepackage{hyperref}

\usepackage{abstract}
\usepackage{graphicx}
\usepackage{gensymb}
\usepackage{caption}
\usepackage{amsmath}
\usepackage{amsthm}
\usepackage{amsfonts}
%\usepackage{float}
\usepackage{sidecap}
\usepackage{mathtools}
\usepackage{adjustbox}
\usepackage{ upgreek }




\usepackage{epstopdf}%This line makes .eps figures into .pdf - please comment out if not required.

\definecolor{cream}{RGB}{222,217,201}

\begin{document}

\pagestyle{fancy}
\thispagestyle{plain}
\fancypagestyle{plain}{
%%%HEADER%%%
\renewcommand{\headrulewidth}{0pt}
}
%%%END OF HEADER%%%

%%%PAGE SETUP - Please do not change any commands within this section%%%
\makeFNbottom
\makeatletter
\renewcommand\LARGE{\@setfontsize\LARGE{15pt}{17}}
\renewcommand\Large{\@setfontsize\Large{12pt}{14}}
\renewcommand\large{\@setfontsize\large{10pt}{12}}
\renewcommand\footnotesize{\@setfontsize\footnotesize{7pt}{10}}
\makeatother

\renewcommand{\thefootnote}{\fnsymbol{footnote}}
\renewcommand\footnoterule{\vspace*{1pt}% 
\color{cream}\hrule width 3.5in height 0.4pt \color{black}\vspace*{5pt}} 
\setcounter{secnumdepth}{5}

\makeatletter 
\renewcommand\@biblabel[1]{#1}            
\renewcommand\@makefntext[1]% 
{\noindent\makebox[0pt][r]{\@thefnmark\,}#1}
\makeatother 
\renewcommand{\figurename}{\small{Fig.}~}
\sectionfont{\sffamily\Large}
\subsectionfont{\normalsize}
\subsubsectionfont{\bf}
\setstretch{1.125} %In particular, please do not alter this line.
\setlength{\skip\footins}{0.8cm}
\setlength{\footnotesep}{0.25cm}
\setlength{\jot}{10pt}
\titlespacing*{\section}{0pt}{4pt}{4pt}
\titlespacing*{\subsection}{0pt}{15pt}{1pt}
%%%END OF PAGE SETUP%%%

%%%FOOTER%%%
\fancyfoot{}
\fancyfoot[LO,RE]{\vspace{-7.1pt}\includegraphics[height=9pt]{head_foot/LF}}
\fancyfoot[CO]{\vspace{-7.1pt}\hspace{13.2cm}\includegraphics{head_foot/RF}}
\fancyfoot[CE]{\vspace{-7.2pt}\hspace{-14.2cm}\includegraphics{head_foot/RF}}
\fancyfoot[RO]{\footnotesize{\sffamily{1--\pageref{LastPage} ~\textbar  \hspace{2pt}\thepage}}}
\fancyfoot[LE]{\footnotesize{\sffamily{\thepage~\textbar\hspace{3.45cm} 1--\pageref{LastPage}}}}
\fancyhead{}
\renewcommand{\headrulewidth}{0pt} 
\renewcommand{\footrulewidth}{0pt}
\setlength{\arrayrulewidth}{1pt}
\setlength{\columnsep}{6.5mm}
\setlength\bibsep{1pt}
%%%END OF FOOTER%%%

%%%FIGURE SETUP - please do not change any commands within this section%%%
\makeatletter 
\newlength{\figrulesep} 
\setlength{\figrulesep}{0.5\textfloatsep} 

\newcommand{\topfigrule}{\vspace*{-1pt}% 
\noindent{\color{cream}\rule[-\figrulesep]{\columnwidth}{1.5pt}} }

\newcommand{\botfigrule}{\vspace*{-2pt}% 
\noindent{\color{cream}\rule[\figrulesep]{\columnwidth}{1.5pt}} }

\newcommand{\dblfigrule}{\vspace*{-1pt}% 
\noindent{\color{cream}\rule[-\figrulesep]{\textwidth}{1.5pt}} }

\makeatother
%%%END OF FIGURE SETUP%%%

%%%TITLE, AUTHORS AND ABSTRACT%%%
\twocolumn[
  \begin{@twocolumnfalse}
{\includegraphics[height=30pt]{head_foot/journal_name}\hfill\raisebox{0pt}[0pt][0pt]{\includegraphics[height=55pt]{head_foot/RSC_LOGO_CMYK}}\\[1ex]
\includegraphics[width=18.5cm]{head_foot/header_bar}}\par
\vspace{1em}
\sffamily
\begin{tabular}{m{4.5cm} p{13.5cm} }

\includegraphics{head_foot/DOI} & \noindent\LARGE{\textbf{A High-Throughput Computational Dataset of Halide Perovskite Alloys$^\dag$}} \\%Article title goes here instead of the text "This is the title"
\vspace{0.3cm} & \vspace{0.3cm} \\

 & \noindent\large{Jiaqi Yang,\textit{$^{a}$} Panayotis Manganaris\textit{$^{a}$} and Arun Mannodi-Kanakkithodi\textit{$^{a}$}} \\%Author names go here instead of "Full name", etc.

\includegraphics{head_foot/dates} & \noindent\normalsize{Novel halide perovskites with improved stability and optoelectronic properties can be designed via composition engineering at cation and/or anion sites. Data-driven methods, especially high-throughput first principles computations and subsequent analysis based on unique materials descriptors, are key to achieving this goal. In this work, we report a density functional theory (DFT)-based dataset of 550 ABX$_3$ halide perovskite compounds, with various atomic and molecular species considered at A, B and X sites, and different amounts of mixing considered at each site to simulate special quasirandom structures of alloys. We perform GGA-PBE calculations on pseudo-cubic perovskite structures to determine their lattice constants, stability in terms of formation and decomposition energies, electronic band gaps, and properties extracted from optical absorption spectra. To elucidate the importance of the level of theory used, we further perform 300 calculations using the more expensive HSE06 functional and determine lattice constant, stability and band gap, and compare PBE and HSE06 properties with some experimentally measured results. Trends in the datasets are unraveled in terms of the effects of mixing at different sites, the composition in terms of specific atomic or molecular species, and averaged elemental properties of species at different sites. This work presents the most comprehensive DFT perovskite alloy dataset to date and the data, which is open-source, can be exploited to train predictive and optimization models for accelerating the design of completely new compositions that may yield large solar cell efficiencies and improved performance across many optoelectronic applications.} 

\end{tabular}

 \end{@twocolumnfalse} \vspace{0.6cm}

  ]
%%%END OF TITLE, AUTHORS AND ABSTRACT%%%

%%%FONT SETUP - please do not change any commands within this section
\renewcommand*\rmdefault{bch}\normalfont\upshape
\rmfamily
\section*{}
\vspace{-1cm}


%%%FOOTNOTES%%%

\footnotetext{\textit{$^{a}$~School of Materials Engineering, Purdue University, West Lafayette, IN 47907, USA; E-mail: amannodi@purdue.edu }}

\footnotetext{\dag~Electronic Supplementary Information (ESI) available: [details of any supplementary information available should be included here]. See DOI: 00.0000/00000000.}


%%%END OF FOOTNOTES%%%

%%%MAIN TEXT%%%%


\section*{Introduction}

... \\


\begin{figure*}[h]
\centering
\includegraphics[width=0.99\linewidth]{Figure1.png}
\caption{\label{Fig:outline} 
(a) Chemical space of ABX$_3$ perovskites. (b) Pie charts showing the amounts of various atomic and molecular species as well as types of mixing in the DFT dataset. (c) Detailed outline of this work.}
\end{figure*}


\begin{figure*}[h]
\centering
\includegraphics[width=0.99\linewidth]{Figure2.png}
\caption{\label{Fig:outline} 
Visualization of DFT data: PBE and HSE properties; lattice constants, decomposition energies, band gaps, photovoltaic figure of merit.}
\end{figure*}


\begin{figure*}[h]
\centering
\includegraphics[width=0.80\linewidth]{./composition_kde.png}
\caption{\label{fig:kdes}
  Kernel Density Estimations with respect to main alloy classes of
  Perovskite Compositions }
\end{figure*}


\begin{figure*}[h]
\centering
\includegraphics[width=0.80\linewidth]{Figure3.pdf}
\caption{\label{Fig:outline} 
Comparison of PBE and HSE computed properties with collected experimental measurements: (a) pseudo-cubic lattice constants, and (b) electronic band gaps.}
\end{figure*}


\begin{figure*}[h]
\centering
\includegraphics[width=0.99\linewidth]{Figure4.png}
\caption{\label{Fig:outline} 
Pearson linear correlation coefficients between 50 composition and elemental descriptors and (a) 6 PBE computed properties, and (b) 4 HSE computed properties.}
\end{figure*}


\begin{figure*}[h]
\centering
\includegraphics[width=0.80\linewidth]{Figure5.pdf}
\caption{\label{Fig:outline} 
Data visualization / dimensionality reduction / importance of features / trends in data using PCE, MDS, T-SNE, Isomap, etc.}
\end{figure*}


\begin{figure*}[h]
\centering
\includegraphics[width=0.99\linewidth]{Figure6.png}
\caption{\label{Fig:outline} 
Computed band structure, density of states, and optical absorption spectrum of 6 selected promising mixed composition perovskites.}
\end{figure*}


\newpage



\section*{Methodology}
% \begin{equation}\label{eqn-0}
% \begin{multlined}
% FOM = \sum_{\lambda_i} \alpha(\lambda_i) * I_s(\lambda_i) * (\lambda_{i+1} - \lambda_i) / \sum_{\lambda_i} I_s(\lambda_i) * (\lambda_{i+1} - \lambda_i)
% \end{multlined}
% \end{equation}
\subsection*{Building Perovskite Dataset}
Firstly, we start with known pure perovskite structures from public data base. 90 pure perovskites are constructed and DFT optimization are applied to get relaxed structures. Based on the pure perovskite structures, we add mixing for A, B, and X sites. To create mixing, for example in B sites, we first select possible B sites mixing elements. Then SQS methods are applied to generate random doping sties for given mixing concentrations. Thus, we constructed 126 A-site mixing samples, 151 B-site mixing samples and 127 X-site mixing samples. All samples are optimized by DFT calculations.

\subsection*{DFT Calculation Details}
DFT calculations are performed with Vienna Ab initio Simulation Package (VASP) version 5.4. The projector augmented wave (PAW) potentials were used The generalized gradient approximation (GGA) of Perdew, Burke and Ernzerhof (PBE) is used as exchange-correlation energy. The energy cutoff for the plane-wave basis is set to 500 eV. The Brillouin zone was sampled by Monkhorst-Pack k-point mesh, with a reciprocal mesh as 3x3x3. The structural force convergence threhold is set to be 0.025 eV/Å.

\subsection*{Properties Analysis}
\subsubsection*{Decomposition Energy}
The decomposition energy is an important property showing the stability of perovskite. To calculate the decomposition energy for ABX3 perovskite, we assume it will decompose to two phases, AX and BX2. Using DFT calculations, we can get the optimized energy of perovskite and its decomposed compounds.
%equation for decomposition energy
The equation is presenting the details calculation for decomposition energy. For PBE decomposition energy, we use DFT energy results from PBE level calculation. For HSE decomposition energy, we use the energy from HSE level calculations.
\subsubsection*{Band Gap and Band Structure}

\subsubsection*{LOPTICS Calculations and Absorption Spectrum}

\subsubsection*{SLME Package}

\subsubsection*{Dielectric Constant }

\newpage




\section*{Results and discussion}

\subsection*{Visualization of DFT Data}

Fig 2 shows plots for critical properties of perovskites in the datasets.

\subsubsection*{Lattice constant PBE vs HSE}
Fig 2(a) presents the lattice parameter comparison of PBE calculation and HSE calculations. Most of the samples have close HSE lattice parameters and PBE lattice parameters. This indicates that the accuracy of PBE on lattice and structure is enough for most of the perovskite samples. Some samples are observed significantly different lattice parameters. By checking the optimized structures, we found that the perovskite structure is largely deformed and no longer keep the octahedral structures. Thus, we removed these 3 points for following analysis.
\subsubsection*{Decomposition energy vs band gap PBE & HSE}
Fig 2(b) is showing the PBE band gap compared to the PBE decomposition energy. It presents the diversity of our perovskite dataset. The data sets cover a large range of band gap and decomposition energy. For example, we have perovskite with low decomposition energy (good stability) and suitable band gap value (between 1 eV to 2.5 eV for PBE calculations). And we can also find samples with low stability and large band gap. The distribution of band gap and decomposition energy shows a great diversity of all perovskite samples and indicates that our data set can statistically represent a sufficient perovskite space.
Fig 2(c) shows the plot of HSE band gaps and HSE decomposition energy. Since we applied HSE calculations for part of the samples, it shows some grouping on low decomposition energy. Compared to PBE plot, more data should be added in high decomposition energy region and suitable band gap region.
\subsubsection*{Spectroscopic Limited Maximum Efficiency (SLME) vs PBE Band gap}
Fig 2(d) presents the Spectroscopic Limited Maximum Efficiency (SLME) values related to the PBE band gap. Spectroscopic Limited Maximum Efficiency (SLME) is a very important properties for photovoltaic performance. SLME measures the absorption efficiency of light for the perovskite. As Fig 2(d) showing, a peak around 1.5 eV is obvious. The peak indicates that these samples with 1.5 eV PBE band gap will also have best absorption efficiency as photovoltaic materials. As the band gap increases, the SLME value decreases and eventually goes to zero due to the high band gap values.
\subsection*{Pearson Correlation Results}

\subsection*{PCA / MDS / T-SNE / Isomap Results}

\subsection*{Screening process of all samples in database}
A high-throughput DFT calculation leads to a sufficient perovskite database. From the database, we can screen over all samples with certain properties and obtain some promising perovskites for further DFT calculations and experiments. A great perovskite material should have low deformation level, strong stability, and proper band gap. To fully screen our database, we will investigate the deviation of cubicity, octahedral factor, tolerance factor, Bartel tolerance factor, decomposition energy, and band gap. The following sections will discuss the details of each aspect.
\subsubsection*{Deviation of Cubicity}
For all the compositions we tested in our data set, some of them will have large strain and deformation because of the combination of elements. For these largely deformed samples, they are no longer remain a cubic perovskite structure. In this section, we want to analyst how the structure is apart from the cubic perovskite structure and rule out these largely deformed samples.
Firstly, we need to define deviation of cubicity. For each perovskite sample, we measure the difference between b, c lattice parameter against a lattice, showing in Equation X. If the lattice deviation of cubicity is larger than 10\%, we will consider this perovskite is no longer remain a cubic perovskite and exclude it.
Similarly, we also need to consider 3 angles, α, β and γ, to make sure the angle also remain 90 degrees. We calculated the difference of α, β and γ versus 90 degrees, showing in Equation X, and take this as angular deviation of cubicity. We also consider the samples that have more than 5\% of angular deviation of cubicity are not cubic perovskites and exclude them.
Fig 7 (d) shows the screening results for α angle deviation of cubicity. SI Fig X shows the screening results for b, c lattice and β, γ angle. Most of the samples with non-cubic perovskites structures are organic-inorganic hybrid perovskites. A large part of the excluded samples are A-mixing hybrid perovskites. It indicates that organic ligands in A site sometimes increase the lattice along some direction and make the perovskite deformed.
\subsubsection*{Octahedral factor, tolerance factor, and Bartel tolerance factor}
The stability of perovskite can be predicted by using the atom radius of all components. There are 3 types of factors are usually considered, Octahedral factor, tolerance factor, and Bartel tolerance factor. The formula of these three factors are shown in equation XXX. In our screening process, we set the criteria for Octahedral factor as 0.442 – 0.895. The criteria for tolerance factor is set to be 0.813 – 1.107. The criteria for Bartel tolerance factor is set to be less than 4.18.
Fig 7 (a) shows the Octahedral factor versus decomposition energy plot. Fig 7 (b) shows the tolerance factor versus decomposition energy plot. The tolerance factor shows a trend that as the tolerance factor increases, the decomposition energy decreases. Fig 7 (c) shows the Bartel tolerance factor versus decomposition energy plot. Within the criteria of Bartel tolerance factor, the decomposition energy is rapidly decreased.
\subsubsection*{Screening Results}




... \\

% \begin{table}
% \centering
%   \caption{\ NN model training and test prediction RMSEs for every property.}
%   \label{table:rmse}
%   \begin{tabular}{ccc}
%     \hline
%   &   &   \\
% \textbf{Property}  &  \textbf{Training Set RMSE}  &  \textbf{Test Set RMSE} \\
%   &   &   \\
% \hline
%   &   &   \\
%       PBE Lattice Constant  & 0.09 \AA  & 0.10 \AA  \\
%       HSE Lattice Constant  & 0.06 \AA  & 0.06 \AA  \\
%       $\Delta$H$_{decomp}$ (PBE) & 0.05 eV  & 0.11 eV  \\
%       $\Delta$H$_{decomp}$ (HSE)  & 0.05 eV  & 0.09 eV  \\
%       E$_{gap}$$^{PBE}$  & 0.20 eV  & 0.22 eV  \\
%       E$_{gap}$$^{HSE}$  & 0.19 eV  & 0.24 eV  \\
%       Refractive Index  & 0.04  & 0.05  \\
%       PV FOM (log$_{10}$)  & 0.14  & 0.18  \\
%       X-rich D.F.E.  & 0.12 eV  & 0.23 eV  \\
%       X-rich E$_{F}$  & 0.06 eV  & 0.19 eV  \\
%       Medium-X D.F.E.  & 0.18 eV  & 0.29 eV  \\
%       Medium-X E$_{F}$  & 0.11 eV  & 0.30 eV  \\
%       B-rich D.F.E.  & 0.11 eV  & 0.30 eV  \\
%       B-rich E$_{F}$  & 0.11 eV  & 0.25 eV  \\
%       V$_{A}$ (0/-1) & 0.07 eV  & 0.11 eV  \\
%       V$_{X}$ (+1/0)  & 0.19 eV  & 0.22 eV  \\
%   &   &   \\
%     \hline
%   \end{tabular}
% \end{table}

\newpage



\section*{Perspective and Future Work}
    
... \\



\section*{Conclusions}

... \\


\section*{Conflicts of interest}
There are no conflicts to declare.

\section*{Acknowledgements}
Extensive discussions with and scientific feedback from UC San Diego researchers David Fenning and Rishi Kumar and Argonne National Lab scientist Maria Chan are acknowledged. This work was performed at Purdue University, under startup account F.10023800.05.002 from the Materials Engineering department. This research used resources of the National Energy Research Scientific Computing Center, the Laboratory Computing Resource Center at Argonne National Laboratory, and the RCAC clusters at Purdue.


\balance


\bibliography{rsc} %You need to replace "rsc" on this line with the name of your .bib file
\bibliographystyle{rsc} %the RSC's .bst file





\clearpage
\newpage
\setcounter{page}{1}

%\onecolumngrid

\setcounter{figure}{0}   
\setcounter{table}{0} 
\renewcommand{\thetable}{S\Roman{table}} 
\renewcommand\thefigure{S\arabic{figure}}
 
\begin{center}
\vspace*{0.5cm}
\Large
\textbf{Supplemental material to "A High-Throughput Computational Dataset of Halide Perovskite Alloys"\\}
\vspace{0.5cm}
\large
Jiaqi Yang,\textit{$^{a}$} Panayotis Manganaris\textit{$^{a}$} and Arun Mannodi-Kanakkithodi\textit{$^{a}$} \\
\vspace{0.3cm}

\normalsize
\textsuperscript{a}\textit{School of Materials Engineering, Purdue University, West Lafayette, Indiana 47907, USA} \\
\end{center}

\footnote{
\textsuperscript{a}amannodi@purdue.edu\hspace{0.3cm}}

\vspace{1cm}



% \begin{figure*}[h]
% \centering
% \includegraphics[width=\linewidth]{SI_figs/Table_elem_prop.pdf}
% \caption{\label{Fig:SI_elem_prop} 
% List of 15 elemental/molecular properties and tabulated values used for each A, B and X-site constituent in the halide perovskite chemical space.}
% \end{figure*}


% \begin{table*}[h]
% \centering
%   \caption{\ Calculated PV figures of merit (in log$_{10}$) for known semiconductors and some selected compounds from the current study.}
%   \label{table:SI_FOM}
%   \begin{tabular}{cc}
%     \hline
%   &   \\
% \textbf{Compound}  &  \textbf{log$_{10}$ (PV FOM)} \\
%   &   \\
% \hline
%   &   \\
% Si    &   5.64   \\
% SiC   &   3.76   \\
% GaAs   &   5.66   \\
% CdTe   &   5.46   \\
% CdSe   &   5.31   \\
% 	   &   \\
% CsCa$_{0.25}$Ba$_{0.125}$Ge$_{0.25}$Pb$_{0.375}$Cl$_{3}$   &   4.72   \\
% MACa$_{0.375}$Ge$_{0.5}$Pb$_{0.125}$Br$_{3}$   &   4.88   \\
% MACa$_{0.125}$Sr$_{0.125}$Ba$_{0.125}$Ge$_{0.125}$Pb$_{0.5}$I$_{3}$   &   5.07   \\
% CsCa$_{0.875}$Pb$_{0.125}$Br$_{3}$   &   3.84   \\
% FACa$_{0.125}$Sr$_{0.375}$Ba$_{0.125}$Sn$_{0.125}$Pb$_{0.25}$I$_{3}$   &   4.88   \\
%   &   \\
%     \hline
%   \end{tabular}
% \end{table*}

\end{document}
