% Created 2022-01-27 Thu 14:17
% Intended LaTeX compiler: pdflatex
\documentclass[9pt, compress]{beamer}
\usepackage[utf8]{inputenc}
\usepackage[T1]{fontenc}
\usepackage{graphicx}
\usepackage{longtable}
\usepackage{wrapfig}
\usepackage{rotating}
\usepackage[normalem]{ulem}
\usepackage{amsmath}
\usepackage{amssymb}
\usepackage{capt-of}
\usepackage{hyperref}
\institute[Mannodi Group]{Mannodi Group - Purdue MSE}
\mode<beamer>{\usetheme{Warsaw}}
\useoutertheme{miniframes}
\usetheme{default}
\author{Panayotis Manganaris\inst{1}}
\date{\today}
\title{Figure Outline for Computational Perovskite Alloys Dataset}
\hypersetup{
 pdfauthor={Panayotis Manganaris\inst{1}},
 pdftitle={Figure Outline for Computational Perovskite Alloys Dataset},
 pdfkeywords={},
 pdfsubject={},
 pdfcreator={Emacs 27.2 (Org mode 9.5)}, 
 pdflang={English}}
\begin{document}

\maketitle
\begin{frame}{Outline}
\tableofcontents
\end{frame}

\expandafter\def\expandafter\insertshorttitle\expandafter{%
  \insertshorttitle\hfill
  \insertframenumber\,/\,\inserttotalframenumber}
\section{Methodology}
\label{sec:org9cf44e5}
\begin{frame}[allowframebreaks]{DFT simulation premise}
\begin{columns}
\begin{column}{0.5\columnwidth}
\begin{block}{Perovskite structure summary}
\begin{figure}[htbp]
\centering
\includegraphics[width=100]{Methodology/2022-01-24_19-23-38_screenshot.png}
\caption{\label{fig:struct} ABX\textsubscript{3} Cubic Perovskite Structure}
\end{figure}
\end{block}
\end{column}
\begin{column}{0.5\columnwidth}
\begin{block}{Perovskite Chemical Domain}
\begin{table}[htbp]
\caption{\label{tab:org596eb89}\label{tbl:chem} ABX\textsubscript{3} Chemical Domain}
\centering
\begin{tabular}{lll}
A-site & B-site & X-site\\
\hline
MA & Pb & I\\
FA & Sn & Br\\
Cs & Ge & Cl\\
Rb & Ba & \\
K & Sr & \\
 & Ca & \\
 & Be & \\
 & Mg & \\
 & Si & \\
 & V & \\
 & Cr & \\
 & Mn & \\
 & Fe & \\
 & Ni & \\
 & Zn & \\
 & Pd & \\
 & Cd & \\
 & Hg & \\
\end{tabular}
\end{table}
\end{block}
\end{column}
\end{columns}
\end{frame}

\begin{frame}[allowframebreaks]{Composition Space Sampling}
\begin{block}{construction of simulations}
\begin{table}[htbp]
\caption{\label{tbl:mixing} Mix Table}
\centering
\begin{tabular}{lr}
cell construct & trials\\
\hline
2x2x2 Supercell A-site mixed & 126\\
2x2x2 B \& X-site mixed & 5\\
2x2x2 Supercell B-site mixed & 151\\
3x3x3 Supercell B-site mixed & 5\\
4x4x4 Supercell B-site mixed & 10\\
Alternative B-site elements & 36\\
2x2x2 Pure & 90\\
2x2x2 X-site mixed & 127\\
\end{tabular}
\end{table}
\end{block}
\begin{block}{Sampling in DFT dataset}
\begin{figure}[htbp]
\centering
\includegraphics[width=225]{./.ob-jupyter/8f1d4f1aa030bd0ee679d35e9f028fd0a4997cc5.png}
\caption{\label{fig:dft_rep} Species by weight and frequency, and alloy representations in DFT dataset}
\end{figure}
\end{block}
\begin{block}{Sampling in experimental dataset}
data sourced from \cite{almora-2020-devic-perfor}.
\begin{figure}[htbp]
\centering
\includegraphics[width=225]{./.ob-jupyter/dcb62144b4024f99ad012a5dac7e0bdac84b5968.png}
\caption{\label{fig:exp_rep} Species by weight and frequency, and alloy representations in experimental dataset to date}
\end{figure}
\end{block}
\end{frame}
\begin{frame}[allowframebreaks]{Topology of Computational Composition Space}
\begin{block}{PCA projection of Mannodi compositions}
\begin{figure}[htbp]
\centering
\includegraphics[width=160]{./.ob-jupyter/5902b5ff1d6349e8f36ce1aabaede144c396a274.png}
\caption{Definite Prismatic Topology in the Chemical Sample}
\end{figure}
\end{block}
\begin{block}{computation samples Variance shares}
\begin{figure}[htbp]
\centering
\includegraphics[width=200]{./.ob-jupyter/8b50fbbee4a1d9a4c730d46e9dca6d5e2495cf26.png}
\caption{Variance in chemical ratios is fairly evenly spread. So, we expect little domain bias in future model training.}
\end{figure}
\end{block}
\end{frame}
\begin{frame}[allowframebreaks]{Topology of Experimental Composition Space}
\begin{block}{projection of Experimental compositions into Mannodi space}
\begin{figure}[htbp]
\centering
\includegraphics[width=160]{./.ob-jupyter/716a588caef9af7978ee203e8bc1d6a9b8ca3274.png}
\caption{experimental data currently covers only boundaries of experimental domain. Alloying is more thoroughly explored in experimental domain.}
\end{figure}
\end{block}
\begin{block}{Experimental samples Variance shares}
\begin{figure}[htbp]
\centering
\includegraphics[width=200]{./.ob-jupyter/7fa258949ef6f94fc65290b3f795348b6257ca80.png}
\caption{Variance in chemical ratios remains even.}
\end{figure}
\end{block}
\end{frame}
\section{Results}
\label{sec:orgc210d14}
\begin{frame}[allowframebreaks]{Computational vs Experimental}
\begin{block}{Band Gaps}
\begin{figure}[htbp]
\centering
\includegraphics[width=225]{./.ob-jupyter/BGcorrob.png}
\caption{\label{fig:bg_corr} HSE and PBE bandgaps vs experimental measures show clearly computation methods need improvement}
\end{figure}
\end{block}
\end{frame}
\begin{frame}[allowframebreaks]{Trends in Computational Data}
\begin{block}{LC vs Bg in HSE results}
\begin{figure}[htbp]
\centering
\includegraphics[width=220]{./.ob-jupyter/7d2d75ceef4bb57614094fcb0a2af8b9d15918ca.png}
\caption{\label{fig:HSE_clust} In this projection larger lattice constants appear to inversely correlate with larger band gaps}
\end{figure}
\end{block}
\begin{block}{SLME vs Bg in PBE results}
\begin{figure}[htbp]
\centering
\includegraphics[width=220]{./.ob-jupyter/31e788491b6aa6d149463dbbcfe5ba03a59088da.png}
\caption{\label{fig:PBE_clust} In this projection larger band gaps appear to inversely correlate with higher SLME values recorded for 5um absorption layers}
\end{figure}
\end{block}
\end{frame}
\section{Modeling:}
\label{sec:org553edfb}
\begin{frame}[allowframebreaks]{Directive:}
Try to find some reliable clustering in crystal properties/composition
features that predict photovoltaic performance.
\begin{block}{Attempt on Computational data}
\begin{figure}[htbp]
\centering
\includegraphics[width=150]{./.ob-jupyter/c96d1bca37c915325d43c4dbe3809364c6ca3751.png}
\caption{\label{fig:prop_top} Magpie Descriptors hypercube shaded by PV\textsubscript{FOM}}
\end{figure}
\end{block}
\begin{block}{Compare to Experimental data}
\begin{figure}[htbp]
\centering
\includegraphics[width=150]{./.ob-jupyter/fd6a7e6e0596e5443ab99ca2ccf1944b360b5e59.png}
\caption{\label{fig:prop_top} Magpie Descriptors projected onto mannodi properties space and shaded by PCE\%}
\end{figure}
\end{block}
\end{frame}

\begin{frame}[label={sec:orgdebe79a}]{Plans:}
Due to the demonstrable topology in the input spaces sampled by these
experiments, tSNE or U-Map projection techniques will be explored for
possible cluster representations.
\end{frame}
\section{reference}
\label{sec:org1ac50a4}
\begin{frame}[label={sec:org94a0c7d}]{citations}
\bibliographystyle{plain}
\bibliography{../../../../org/bibliotex/bibliotex}
\end{frame}
\end{document}